\documentclass[12pt]{article}

\usepackage{fullpage}
\usepackage{epic}


\newtheorem{statement}{~~~(\hspace*{-4pt}}[section]

\newcommand{\Xomit}[1]{}

\newcommand{\proof}[1]{
{\noindent {\it Proof.} {#1} \rule{2mm}{2mm} \vskip \belowdisplayskip}
}

\newcommand{\rf}[1]{
%%{{\hspace*{-8pt} (\ref{#1}) \hspace*{-8pt}}}
{{\hspace*{-6pt} (\ref{#1}) \hspace*{-6pt}}}
}

\def\stm{\hspace*{-5pt}{\bf )}~~}

\newcommand{\prevs}[2]{
{\vskip 0.1in \noindent {\em Proof of \rf{#1}.} {#2} \rule{2mm}{2mm}
\vskip \belowdisplayskip}
}

\newcommand{\prevproof}[2]{
{\vskip 0.1in \noindent {\em Proof of {#1}.} {#2} \rule{2mm}{2mm}
\vskip \belowdisplayskip}
}

\newcommand{\St}[1]{
\begin{itemize} \item {{#1}} \end{itemize}
}


\newcommand{\Pa}[1]{
\begin{quote} {\bf ($\dagger$)} {#1} \end{quote}
}

\newcommand{\Pp}[1]{
\begin{quote} {\bf $\triangleright$} {#1} \end{quote}
}

\newcommand{\xhdr}[1]{
\subsection*{#1}
}

\newcommand{\kth}[1]{
{{#1}^{\rm th}}
}

\newcommand{\cel}[1]{
{\lceil {#1} \rceil}
}

\newcommand{\flr}[1]{
{\lfloor {#1} \rfloor}
}

\def\sneq{{\tiny \mbox{$\neq$}}}
\def\subsetneq{\ \lower.5ex\hbox{$\stackrel{\subset}{\sneq}$}\ }

\newcommand{\topmatter}[2]{
\setlength{\fboxrule}{.5mm}\setlength{\fboxsep}{1.2mm}
\newlength{\boxlength}\setlength{\boxlength}{\textwidth}
\addtolength{\boxlength}{-4mm}
\begin{center}\framebox{\parbox{\boxlength}{\bf
Notes on {#1} \hfill {#2} \\ %% Title and Date
Jon Kleinberg and \'Eva Tardos \hfill
}}\end{center}
\vspace{5mm}
}


%%General
\def\ve{{\varepsilon}}
\def\gap{0.2in}

%%stable
\def\GS{{G-S}}

%%greedy
\def\O{{\cal O}}

%%divc
\def\ms{{\sc Mergesort}}
\def\qs{{\sc Quicksort}}

%%dp
\def\seq{{\sigma}}
\def\O{{\cal O}}
\def\P{{\cal P}}
\def\L{{\cal L}}
\def\opt{{OPT}}
\def\v{{value}}

%%flow
\def\O{{\cal O}}
\def\P{{\cal P}}
\def\L{{\cal L}}
\def\opt{{OPT}}

\newcommand{\into}[2]{
{#1}~{\rm into}~{#2}
}

\newcommand{\outof}[2]{
{#1}~{\rm out~of}~{#2}
}

\newcommand{\infl}[2]{
{#1}^{\rm in}({#2})
}

\newcommand{\outfl}[2]{
{#1}^{\rm out}({#2})
}

\def\Re{{\bf R}}
\def\v{{\nu}}

\def\bottle{{\tt bottleneck}}
\def\augment{{\tt augment}}
\def\ffinal{{\overline{f}}}

%%np
\def\IS{{\em Independent Set}}
\def\SP{{\em Set Packing}}
\def\VC{{\em Vertex Cover}}
\def\SC{{\em Set Cover}}
\def\TDM{{\em 3-Dimensional Matching}}
\def\SS{{\em Subset Sum}}
\def\SAT{{\em SAT}}
\def\Sat{{\em Satisfiability}}
\def\TSAT{{\em 3-SAT}}
\def\TSat{{\em 3-Satisfiability}}
\def\HC{{\em Hamiltonian Cycle}}
\def\TS{{\em Traveling Salesman}}
\def\CG{{\em Component Grouping}}

\def\lP{{\leq_P}}
\newcommand{\n}[1]{{\overline{#1}}}

\def\yes{{\tt yes}}
\def\no{{\tt no}}

\def\Po{{\cal P}}
\def\NP{{\cal NP}}
\def\coNP{{{\rm co-}\NP}}

\def\C{{\cal C}}

%%pspace
\def\PSP{{\rm PSPACE}}

\def\PL{{\em Planning}}
\def\QSat{{\em Quantified 3-SAT}}
\def\QSAT{{\em QSAT}}
\def\CSat{{\em Competitive 3-SAT}}
\def\CSAT{{\em CSAT}}
\def\CFL{{\em Competitive Facility Location}}
\def\GG{{\em Geography}}

\def\C{{\cal C}}
\def\O{{\cal O}}
\def\G{{\cal G}}
\def\V{{\cal V}}
\def\E{{\cal E}}

\def\poly{{\rm poly}}
\def\Path{{\tt Path}}

%%approx
\def\LP{{\em Linear Programming}}
\def\IP{{\em Integer Programming}}
\def\ZIP{{\em 0-1 Integer Programming}}

%%exp
\def\setminus{{-}}

\def\opti{{OPT_{in}}}
\def\opto{{OPT_{out}}}
\def\Mi{{M_{in}}}
\def\Mo{{M_{out}}}

\def\MVC{{\em Minimum-Cost Vertex Cover}}
\def\MWIS{{\em Maximum-Weight Independent Set}}

\newcommand{\subgr}[2]{
{#1}_{#2}
}



\begin{document}

Consider the following variation on
the {\em Interval Scheduling Problem} from lecture.
You have a processor that can operate 24 hours a day, every day.
People submit requests to run {\em daily jobs} on the processor.
Each such job comes with a
{\em start time} and an {\em end time};
if the job is accepted to run on the processor, it must
run continuously, every day, for the period between
its start and end times.
(Note that certain jobs can begin before midnight and end
after midnight; this makes for a type of situation different
from what we saw in the Interval Scheduling Problem.)

Given a list of $n$ such jobs, your goal is to accept
{\em as many jobs as possible} (regardless of their length),
subject to the constraint that the processor can run
at most one job at any given point in time.
Provide an algorithm to do this with a running time
that is polynomial in $n$, the number of jobs.
You may assume for simplicity that no two jobs have the
same start or end times.

{\bf Example:} Consider the following four jobs, specified
by {\em (start-time, end-time)} pairs.
\begin{quote}
(6 pm, 6 am), (9 pm, 4 am), (3 am, 2 pm), (1 pm, 7 pm).
\end{quote}
The unique solution would be to pick the two jobs
(9 pm, 4 am) and (1 pm, 7 pm), which can be scheduled
without overlapping.
\par 
Let $I_1, \ldots, I_n$ denote the $n$ intervals.
We say that an {\em $I_j$-restricted solution} 
is one that contains the interval $I_j$.

Here is an algorithm, for fixed $j$, to compute
an $I_j$-restricted solution of maximum size.
Let $x$ be a point contained in $I_j$.
First delete $I_j$ and all intervals that overlap it.
The remaining intervals do not contain the point $x$,
so we can ``cut'' the time-line at $x$ and produce
an instance of the Interval Scheduling Problem from class.
We solve this in $O(n)$ time, assuming that the intervals
are ordered by ending time.

Now, the algorithm for the full problem is to
compute an $I_j$-restricted solution of maximum size
for each $j = 1, \ldots, n$.
This takes a total time of $O(n^2)$.
We then pick the largest of these solutions, and
claim that it is an optimal solution.
To see this, consider the optimal solution to 
the full problem, consisting of a set of intervals $S$.
Since $n > 0$, there is some interval $I_j \in S$;
but then $S$ is an optimal $I_j$-restricted solution,
and so our algorithm will produce a solution at 
least as large as $S$.
%%%%%%%%%%%%%%%%%
\end{document}


